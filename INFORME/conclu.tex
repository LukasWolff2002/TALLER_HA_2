\section{Conclusiones}

En conclusión, este taller buscó realizar un estudio del análisis estadístico de la resistencia mecánica del hormigón extraído de obras, aplicando las normas NCh1998 y NCh1171-1-2001 para evaluar la conformidad de los resultados obtenidos. A partir de los datos de entrada, se determinó que el hormigón no cumple la resistencia especificada, lo que implica un riesgo estructural y la necesidad de adoptar medidas correctivas. Asimismo, se evaluó el nivel de control de los ensayos, obteniéndose un nivel excelente, lo que indica un adecuado proceso de muestreo y una baja variabilidad entre probetas.

Además, se discutió qué normas prescriptivas y de desempeño pueden aplicarse en distintas situaciones, en conexión con el desarrollo del taller anterior, en el que se llevó a cabo una optimización granulométrica para el hormigón solicitado.

Finalmente, se generaron diversas discusiones sobre los efectos de distintos factores en la mezcla de hormigón, como la adición de productos químicos y la porosidad, entre otros, lo que permitió no solo analizar las muestras por sus resultados mecánicos, sino también alcanzar una comprensión más amplia de los múltiples elementos que pueden estar involucrados.