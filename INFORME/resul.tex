\section{Desarrollo}

\subsection{Metodología}

\subsubsection{Análisis estadístico de la resistencia mecánica}

En primer lugar, se realizó un análisis estadístico de la resistencia mecánica, basado en la norma \cite{NCh1998}, cuyo objetivo es determinar la conformidad de los resultados de la resistencia real a compresión respecto de la especificada.

Para poder aplicar la norma, se deben cumplir los siguientes requisitos básicos:

\begin{itemize}
    \item Toma de muestras al azar.
    \item Toma de muestras entre el 10\% y el 90\% de la descarga del camión mixer.
    \item Tamaño de muestra superior a 1,5 veces el necesario.
    \item Generar al menos 3 probetas por muestra: una ensayada a 7 días y las restantes a 28 días.
\end{itemize}

La Tabla \ref{tab:muestreo} detalla el número y volumen de muestras según la procedencia del hormigón.

\begin{table}[H]
\centering
\begin{tabular}{|l|c|c|}
\hline
\textbf{Procedencia del hormigón} & \multicolumn{2}{c|}{\textbf{Volumen de hormigón de la obra, m$^3$}} \\ \hline
                                  & $> 250$ & $\leq 250$ \\ \hline
\textbf{Fabricado en obra}        &         &            \\ \hline
Volumen máximo de hormigón por muestra & 100     & 50        \\ \hline
Número mínimo de muestras              & 5       & 3         \\ \hline
\textbf{De central hormigonera}   &         &            \\ \hline
Volumen máximo de hormigón por muestra & 150     & 75        \\ \hline
Número mínimo de muestras              & 5       & 3         \\ \hline
\end{tabular}
\caption{Requisitos de muestreo según la procedencia y el volumen de hormigón.}
\label{tab:muestreo}
\end{table}

Además, el plan de muestreo debe estar especificado en la norma de diseño o en la especificación de la obra. En caso contrario, se debe considerar lo siguiente:

\begin{itemize}
    \item Pavimentos: un testigo cada 1000 m$^2$, con un mínimo de 3.
    \item General: testigos por zona.
\end{itemize}

Si se dispone de un número $N$ mayor a 10 muestras, se puede considerar que la resistencia de cada lote es satisfactoria si se cumple lo siguiente:

\begin{itemize}
    \item Criterio global: $f_3 \geq f_c + k_1$
    \item Criterio individual: $f_i \geq f_0 = f_c - k_2$
\end{itemize}

Donde $f_3$ corresponde a la resistencia media de tres muestras consecutivas a 28 días, $f_c$ es la resistencia especificada, $f_i$ es la resistencia individual de cada muestra y $f_0$ es la resistencia mínima aceptable para cada $f_i$ a 28 días.

Los valores de $k_1$ y $k_2$ se indican en la Tabla \ref{tab:valoresk}.

\begin{table}[H]
\centering
\begin{tabular}{|c|c|c|c|c|c|}
\hline
\textbf{Fracción defectuosa aceptada, \%} &  & \multicolumn{4}{c|}{\textbf{Grado de hormigón}} \\ \hline
 &  & H5 & H10 & H15 & H20 o superior \\ \hline
\multirow{2}{*}{5}  & $k_{1}$ & 0,3 & 0,5 & 0,8 & 1,0 \\ \cline{2-6}
                    & $k_{2}$ & 0,6 & 1,2 & 1,9 & 2,5 \\ \hline
\multirow{2}{*}{10} & $k_{1}$ & 0   & 0   & 0   & 0   \\ \cline{2-6}
                    & $k_{2}$ & 0,9 & 1,7 & 2,6 & 3,5 \\ \hline
\multirow{2}{*}{20} & $k_{1}$ & 0,4 & 0,7 & 1,1 & 1,5 \\ \cline{2-6}
                    & $k_{2}$ & 1,4 & 2,7 & 4,1 & 5,5 \\ \hline
\end{tabular}
\caption{Valores de $k_{1}$ y $k_{2}$ según la fracción defectuosa aceptada y el grado de hormigón.}
\label{tab:valoresk}
\end{table}

La Tabla \ref{tab:conclusiones} resume las conclusiones posibles en función de los antecedentes.

\begin{table}[H]
\centering
\begin{tabular}{|c|c|p{6cm}|p{6cm}|}
\hline
\textbf{Antecedentes} &  & \textbf{Conclusiones} & \textbf{Recomendaciones} \\ \hline
$f_{3} \geq f_{c} + k_{1}$ & $f_{i} \geq f_{0}$ & 
El hormigón cumple la resistencia especificada & \\ \hline

$f_{3} < f_{c} + k_{1}$ & $f_{i} \geq f_{0}$ & 
El hormigón no cumple la resistencia especificada & 
Informar a los proyectistas estructurales y considerar las penalizaciones establecidas en el contrato y sus documentos anexos. \\ \hline

 & $f_{i} < f_{0}$ & 
El hormigón no cumple la resistencia especificada y cada resultado defectuoso debe ser considerado como un riesgo potencial & 
Adoptar las medidas indicadas en A.4 \\ \hline
\end{tabular}
\caption{Conclusiones y recomendaciones en función de los antecedentes de resistencia del hormigón.}
\label{tab:conclusiones}
\end{table}

Adicionalmente, se puede realizar un análisis considerando el total de muestras. El ensayo se considera satisfactorio si se cumple que:

\begin{itemize}
    \item Criterio global: $f_{m} \geq f_{c} + s \cdot t$
    \item Criterio individual: $f_{i} \geq f_{0} = f_c - k_2$
\end{itemize}

Donde $s$ corresponde a la desviación estándar de la muestra, y el valor $t$ puede calcularse a partir de los datos de la Tabla 4 de la norma \cite{NCh1998}.

La Tabla \ref{tab:conclusiones2} muestra las conclusiones obtenidas en este análisis.

\begin{table}[H]
\centering
\begin{tabular}{|c|c|p{6cm}|p{6cm}|}
\hline
\textbf{Antecedentes} &  & \textbf{Conclusiones} & \textbf{Recomendaciones} \\ \hline
$f_{m} \geq f_{c} + s \cdot t$ & $f_{i} \geq f_{0}$ & 
El hormigón cumple la resistencia especificada & \\ \hline

$f_{m} < f_{c} + s \cdot t$ & $f_{i} \geq f_{0}$ & 
El hormigón no cumple la resistencia especificada & 
Informar a los proyectistas estructurales y considerar las penalizaciones establecidas en el contrato y sus documentos anexos. \\ \hline

 & $f_{i} < f_{0}$ & 
El hormigón no cumple la resistencia especificada y cada resultado defectuoso debe ser considerado como un riesgo potencial & 
Adoptar las medidas indicadas en A.4 \\ \hline
\end{tabular}
\caption{Conclusiones y recomendaciones considerando $f_{m}$, $f_{c}$, $s$, $t$ y $f_{0}$.}
\label{tab:conclusiones2}
\end{table}

Finalmente, si se dispone de por lo menos 10 muestras distitnas, con dos o mas probetas por cada una, se puede realizar una evaluacion del nivel de control de los ensayos.

En primer lugar es nesesario calcular el intervalo promedio R:

\begin{equation}
    \bar{R} = \frac{\sum R_i}{N}
\end{equation}

Donde:

\begin{itemize}
    \item $R_i = f_{max,i} - f_{min,i}$ es el rango de cada muestra.
    \item $N$ es el número total de muestras.
\end{itemize}

A continuación, se debe calcular la desviacion normal de los ensayos:

\begin{equation}
    s_1 = \frac{\bar{R}}{d_2}
\end{equation}

Para el valor de d2, considerar los datos de la tabla \ref{tab:d2}

\begin{table}[H]
\centering
\begin{tabular}{|c|c|}
\hline
\textbf{Número de probetas, $n_{0}$} & \textbf{$d_{2}$} \\ \hline
2 & 1,128 \\ \hline
3 & 1,693 \\ \hline
4 & 2,059 \\ \hline
5 & 2,326 \\ \hline
6 & 2,534 \\ \hline
\end{tabular}
\caption{Valores de $d_{2}$ en función del número de probetas $n_{0}$.}
\label{tab:d2}
\end{table}

Finalmente se calcula el coeficiente de vaiacion de ensayo:

\begin{equation}
    V_1 = \frac{s_1}{f_m} \cdot 100\%
\end{equation}

Donde la tabla \ref{tab:conclusiones3} muestra las conclusiones y recomendaciones en función del coeficiente de variación obtenido.

\begin{table}[H]
\centering
\begin{tabular}{|c|c|}
\hline
\textbf{$V_{1}$ (\%)} & \textbf{Nivel de control de ensayos} \\ \hline
$0 \leq V_{1} \leq 3,0$ & Excelente \\ \hline
$3,0 < V_{1} \leq 4,0$ & Muy bueno \\ \hline
$4,0 < V_{1} \leq 5,0$ & Bueno \\ \hline
$5,0 < V_{1} \leq 6,0$ & Aceptable \\ \hline
$6,0 < V_{1}$ & Deficiente \\ \hline
\end{tabular}
\caption{Clasificación del nivel de control de ensayos según el valor de $V_{1}$.}
\label{tab:conclusiones3}
\end{table}


\subsubsection{Analisis de Testigos}

La norma \cite{NCh1171-1-2001} establece los requicitos de analisis para la extraccion de testigos. Se deben cumplir los siguientes requerimientos:

\begin{itemize}
    \item El diametro debe ser mayor que el doble del diametro de maximo agregado
    \item Altura tal que la esbeltez sea entre 1.0 y 2.0
    \item La resistencia de extraccion debe ser mayor a 8 MPa y mayor a 14 dias.
    \item Ademas, los limites fisicos de extraccion son especificados por la norma.
\end{itemize}












\subsection{Resultados}

Los datos de entrada son un hormigon G55 con un porcentaje de defectuoso del 20\%, ademas, se tienen las siguientes muestras y probetas (tabla \ref{tab:datosentrada}).

\begin{table}[H]
\centering
\begin{tabular}{|c|c|c|c|}
\hline
\textbf{Correlativo} & \textbf{7 días [MPa]} & \textbf{28 días [MPa]} & \textbf{28 días [MPa]} \\ \hline
1  & 41.45 & 52.8  & 54.6  \\ \hline
2  & 40.2  & 57.11 & 56.6  \\ \hline
3  & 43.5  & 52.35 & 55.18 \\ \hline
4  & 41    & 55.25 & 54.2  \\ \hline
5  & 40    & 54.48 & 53.82 \\ \hline
6  & 41    & 55.3  & 52.08 \\ \hline
7  & 43.5  & 51    & 53.82 \\ \hline
8  & 41    & 48.4  & 49.9  \\ \hline
9  & 40    & 49.5  & 45.6  \\ \hline
10 & 40.2  & 52.56 & 47.88 \\ \hline
11 & 43.5  & 57.11 & 56.6  \\ \hline
12 & 40    & 52.35 & 52.08 \\ \hline
13 & 41    & 55.25 & 53.82 \\ \hline
14 & 43.5  & 54.48 & 53.82 \\ \hline
15 & 41    & 55.3  & 57.79 \\ \hline
16 & 40    & 58.15 & 56.6  \\ \hline
17 & 40.2  & 55.25 & 55.18 \\ \hline
18 & 43.5  & 52.56 & 53.82 \\ \hline
19 & 41    & 48.4  & 45.68 \\ \hline
20 & 40    & 49.6  & 47.7  \\ \hline
\end{tabular}
\caption{Resultados de resistencia a compresión de probetas a 7 y 28 días.}
\label{tab:datosentrada}
\end{table}

Luego, se obtienen los siguientes resultados: 

\begin{table}[H]
\centering
\begin{tabular}{|c|c|c|}
\hline
\textbf{Probeta} & \textbf{Cumple $f_c - k_2$} & \textbf{Cumple $f_3 \geq f_c + k_1$} \\ \hline
1  & Sí & Sí \\ \hline
2  & Sí & Sí \\ \hline
3  & Sí & Sí \\ \hline
4  & Sí & Sí \\ \hline
5  & Sí & No \\ \hline
6  & Sí & No \\ \hline
7  & Sí & No \\ \hline
8  & No & No \\ \hline
9  & No & No \\ \hline
10 & Sí & No \\ \hline
11 & Sí & Sí \\ \hline
12 & Sí & Sí \\ \hline
13 & Sí & Sí \\ \hline
14 & Sí & Sí \\ \hline
15 & Sí & Sí \\ \hline
16 & Sí & Sí \\ \hline
17 & Sí & No \\ \hline
18 & Sí & No \\ \hline
19 & No & - \\ \hline
20 & No & - \\ \hline
\end{tabular}
\caption{Resultados de cumplimiento de las condiciones estadísticas de resistencia del hormigón.}
\label{tab:cumplimiento}
\end{table}

Ademas, el criterio $f_m \geq f_c + st$ no se cumple. De esta forma, se puede concluir que es nesesario adoptar las medidas indicadas en A.4. de la norma \cite{NCh1998}.

Se concluye que:

\begin{itemize}
    \item Existe riesgo estructural, y el hormigón afectado debe ser sometido a análisis.
    \item Se debe comprobar la validez del ensayo, además de extraer testigos de hormigón endurecido para identificar las zonas comprometidas y evaluar la resistencia real del material en obra, de acuerdo con lo establecido en la \citep{NCh1171-1-2001,NCh1171-2-2001}.
    \item Es necesario reforzar el control de calidad en obra para evitar este tipo de incumplimientos, asegurando condiciones adecuadas de confección, transporte y curado del hormigón \citep{NCh170-2016}.
\end{itemize}

Adicionalmente, se calcula el nivel de control del ensayo, donde se concluye que la calidad es excelente, lo cual refleja que la variabilidad entre probetas es baja y que el proceso de muestreo fue realizado correctamente. Sin embargo, la resistencia promedio no cumple con el valor exigido, lo que confirma que la causa principal del incumplimiento se debe a la resistencia real del lote de hormigón y no a errores de ensayo.

Al modificar el porcentaje de defectuoso, no se genera ningún cambio al aumentarlo, ya que la dosificación entregada se encuentra en el límite superior (20\%). En cambio, al reducir dicho porcentaje se observa una disminución en el número de muestras aprobadas. Esto se debe a que los factores estadísticos $k$ aumentan, incrementando la diferencia exigida en los criterios de aceptación y, por ende, reduciendo el número de resultados válidos. 

Por ejemplo, con 26 muestras iniciales todas aprobadas bajo el criterio del 20\% de defectuosos, si se restringe el límite a un 5\%, solo 15 probetas cumplen la condición. Esta sensibilidad demuestra la relevancia de definir correctamente la fracción defectuosa en la especificación de obra, pues influye directamente en la aceptación o rechazo de lotes y en la necesidad de aplicar medidas correctivas adicionales \citep{NCh1998}.


\subsection{Discusión}

\begin{enumerate}
    \item El proceso de hidratación del cemento es fundamental en el desarrollo de la resistencia a compresión del hormigón, ya que genera las reacciones químicas entre el agua y los compuestos del cemento, formando compuestos que consolidan la matriz. A los 7 días, se ha alcanzado un porcentaje considerable de hidratación (60–70\%), lo que explica que el hormigón presente una resistencia significativa. A los 28 días se logra la resistencia característica de diseño, mientras que a los 90 días el proceso residual de hidratación incrementa la densificación de la microestructura y la durabilidad. Los factores externos como la temperatura y la humedad afectan la velocidad de estas reacciones: temperaturas altas aceleran la hidratación pero pueden inducir microfisuras y pérdidas de resistencia a largo plazo; temperaturas bajas retardan el proceso. Un curado con adecuada humedad asegura la continuidad de la hidratación, mientras que un curado con baja humedad relentiza el proceso, aumenta la porosidad y reduce la resistencia y la durabilidad \citep{NCh170-2016}.

    \item En relación con la sustentabilidad del hormigón, los aditivos químicos más utilizados incluyen reductores de agua, superplastificantes, incorporadores de aire y adiciones minerales suplementarias (como cenizas volantes o humo de sílice). Estos productos permiten mejorar la trabajabilidad, aumentar la resistencia y reducir la relación agua/cemento, pero también tienen un impacto ambiental asociado a su proceso de producción. Por ejemplo, estudios científicos señalan que la producción de superplastificantes a base de naftaleno formaldehído puede generar aproximadamente entre 50 y 80 kg CO$_2$ eq/tonelada de aditivo \citep{Habert2011,Flower2007}. Al incorporarlos en una dosificación típica, la huella de carbono total de la mezcla se incrementa en una magnitud menor al 1\%, aunque sus beneficios en reducción de agua y durabilidad permiten compensar dicho impacto al extender la vida útil de la estructura. Desde el punto de vista normativo, las especificaciones prescriptivas podrían limitar el uso de estos aditivos indicando cantidades máximas, mientras que las especificaciones por desempeño permitirían regular únicamente el resultado esperado (por ejemplo, la resistencia a 28 días o la permeabilidad máxima), reduciendo su uso a lo estrictamente necesario para cumplir con los criterios de calidad y sostenibilidad \citep{NCh170-2016}.

    \item Los ciclos de hielo-deshielo afectan al hormigón mediante la expansión del agua dentro de los poros capilares cuando se congela, lo que genera presiones internas que producen microfisuras y pérdida progresiva de resistencia y durabilidad. Este fenómeno es particularmente crítico en climas fríos y en estructuras expuestas a sales descongelantes. Una especificación prescriptiva puede exigir un contenido mínimo de aire incorporado (por ejemplo, entre 4\% y 6\%) para mejorar la resistencia al congelamiento, mientras que una especificación por desempeño establecería directamente el requisito de superar ensayos de durabilidad frente a ciclos de congelamiento-descongelamiento. Ambas aproximaciones buscan mitigar el daño, aunque el enfoque por desempeño asegura mayor flexibilidad tecnológica y mejor adaptación a condiciones locales \citep{Powers1954,MehtaMonteiro2014}.

    \item La microestructura del hormigón, determinada por la distribución de poros, la densidad de la matriz y la presencia de microfisuras, influye directamente en su resistencia a compresión y durabilidad. Un hormigón con baja porosidad capilar y una matriz densa permite mayores resistencias mecánicas y menor permeabilidad frente a agentes agresivos. Para evaluar esta microestructura, se pueden emplear diversas técnicas experimentales, entre ellas:  
    \begin{itemize}
        \item Microscopía electrónica de barrido (SEM): permite observar la morfología de los poros, la distribución de productos de hidratación y la presencia de microfisuras, entregando información cualitativa y cuantitativa de la estructura interna \citep{Diamond1999}.  
        \item Porosimetría por intrusión de mercurio (MIP): cuantifica el tamaño y volumen de los poros en la matriz cementicia, lo que permite correlacionar la porosidad total con la resistencia y durabilidad del material \citep{Gallucci2012}.  
    \end{itemize}
    Estas técnicas complementan los ensayos mecánicos tradicionales y permiten comprender la relación entre la microestructura y el desempeño del hormigón en servicio.

    \item La diferencia entre especificaciones prescriptivas y por desempeño es clave en la gestión de calidad del hormigón. Las especificaciones prescriptivas indican cómo se debe fabricar y colocar el material, detallando proporciones, métodos y procesos, asegurando así un control estricto de la producción. En cambio, las especificaciones por desempeño se enfocan en los resultados finales, como resistencia, durabilidad o permeabilidad, sin imponer un procedimiento específico. Este enfoque promueve la innovación y la optimización, siempre que los resultados cumplan con los requisitos funcionales. En la práctica, la combinación de ambas modalidades otorga garantías de calidad en obra y flexibilidad en la producción \citep{NCh170-2016}.

    \item Una especificación técnica de hormigón por desempeño utilizada internacionalmente es la establecida en la norma europea EN 206, la cual define requisitos de durabilidad y resistencia en función de las condiciones de exposición ambiental. Un ejemplo relevante es su aplicación en el puente de Øresund, que conecta Dinamarca y Suecia, expuesto a ambientes marinos severos. En este proyecto se especificó que el hormigón debía cumplir una resistencia característica mínima de 50 MPa, limitar la penetración de cloruros por debajo de un umbral determinado y demostrar un comportamiento satisfactorio en ensayos de ciclos de congelación y deshielo. Esta especificación se considera por desempeño ya que no fija la dosificación ni los materiales específicos, sino que establece criterios medibles de durabilidad y resistencia. Los métodos de evaluación incluyeron ensayos acelerados de penetración de cloruros, determinación de resistividad eléctrica y pruebas de resistencia a la compresión a 28 y 90 días. De esta manera, se garantizó que el hormigón cumpliera con los requisitos de vida útil y seguridad estructural en un ambiente altamente agresivo \citep{EN206,ConcreteSociety2010}.
\end{enumerate}




































