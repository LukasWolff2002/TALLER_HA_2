\section{Desarrollo}

\subsection{Metodología}

\subsubsection{Análisis estadístico de la resistencia mecánica}

En primer lugar, se realizó un análisis estadístico de la resistencia mecánica, basado en la norma \cite{NCh1998}, cuyo objetivo es determinar la conformidad de los resultados de la resistencia real a compresión respecto de la especificada.

Para aplicar la norma, se deben cumplir los siguientes requisitos básicos:

\begin{itemize}
    \item Toma de muestras al azar.
    \item Extracción de muestras entre el 10\% y el 90\% de la descarga del camión mixer.
    \item Tamaño de muestra superior en al menos 1,5 veces al necesario.
    \item Generar al menos 3 probetas por muestra: una ensayada a 7 días y las restantes a 28 días.
\end{itemize}

La Tabla \ref{tab:muestreo} detalla el número y volumen de muestras según la procedencia del hormigón.

\begin{table}[H]
\centering
\begin{tabular}{|l|c|c|}
\hline
\textbf{Procedencia del hormigón} & \multicolumn{2}{c|}{\textbf{Volumen de hormigón de la obra, m$^3$}} \\ \hline
                                  & $> 250$ & $\leq 250$ \\ \hline
\textbf{Fabricado en obra}        &         &            \\ \hline
Volumen máximo de hormigón por muestra & 100     & 50        \\ \hline
Número mínimo de muestras              & 5       & 3         \\ \hline
\textbf{De central hormigonera}   &         &            \\ \hline
Volumen máximo de hormigón por muestra & 150     & 75        \\ \hline
Número mínimo de muestras              & 5       & 3         \\ \hline
\end{tabular}
\caption{Requisitos de muestreo según la procedencia y el volumen de hormigón.}
\label{tab:muestreo}
\end{table}

Además, el plan de muestreo debe estar definido en la norma de diseño o en las especificaciones de la obra. En caso contrario, se debe considerar lo siguiente:

\begin{itemize}
    \item Pavimentos: un testigo cada 1000 m$^2$, con un mínimo de 3.
    \item General: testigos distribuidos por zona.
\end{itemize}

Si se dispone de un número $N$ mayor a 10 muestras, se considera que la resistencia de cada lote es satisfactoria si se cumple lo siguiente:

\begin{itemize}
    \item Criterio global: $f_3 \geq f_c + k_1$
    \item Criterio individual: $f_i \geq f_0 = f_c - k_2$
\end{itemize}

Donde $f_3$ corresponde a la resistencia media de tres muestras consecutivas a 28 días, $f_c$ es la resistencia especificada, $f_i$ la resistencia individual de cada muestra y $f_0$ la resistencia mínima aceptable para cada $f_i$ a 28 días. 

Los valores de $k_1$ y $k_2$ se presentan en la Tabla \ref{tab:valoresk}.

\begin{table}[H]
\centering
\begin{tabular}{|c|c|c|c|c|c|}
\hline
\textbf{Fracción defectuosa aceptada, \%} &  & \multicolumn{4}{c|}{\textbf{Grado de hormigón}} \\ \hline
 &  & H5 & H10 & H15 & H20 o superior \\ \hline
\multirow{2}{*}{5}  & $k_{1}$ & 0,3 & 0,5 & 0,8 & 1,0 \\ \cline{2-6}
                    & $k_{2}$ & 0,6 & 1,2 & 1,9 & 2,5 \\ \hline
\multirow{2}{*}{10} & $k_{1}$ & 0   & 0   & 0   & 0   \\ \cline{2-6}
                    & $k_{2}$ & 0,9 & 1,7 & 2,6 & 3,5 \\ \hline
\multirow{2}{*}{20} & $k_{1}$ & 0,4 & 0,7 & 1,1 & 1,5 \\ \cline{2-6}
                    & $k_{2}$ & 1,4 & 2,7 & 4,1 & 5,5 \\ \hline
\end{tabular}
\caption{Valores de $k_{1}$ y $k_{2}$ según la fracción defectuosa aceptada y el grado de hormigón.}
\label{tab:valoresk}
\end{table}

La Tabla \ref{tab:conclusiones} resume las conclusiones posibles en función de los antecedentes.

\begin{table}[H]
\centering
\begin{tabular}{|c|c|p{6cm}|p{6cm}|}
\hline
\textbf{Antecedentes} &  & \textbf{Conclusiones} & \textbf{Recomendaciones} \\ \hline
$f_{3} \geq f_{c} + k_{1}$ & $f_{i} \geq f_{0}$ & 
El hormigón cumple la resistencia especificada & \\ \hline

$f_{3} < f_{c} + k_{1}$ & $f_{i} \geq f_{0}$ & 
El hormigón no cumple la resistencia especificada & 
Informar a los proyectistas estructurales y considerar las penalizaciones establecidas en el contrato y sus documentos anexos. \\ \hline

 & $f_{i} < f_{0}$ & 
El hormigón no cumple la resistencia especificada y cada resultado defectuoso debe ser considerado como un riesgo potencial & 
Adoptar las medidas indicadas en A.4 \\ \hline
\end{tabular}
\caption{Conclusiones y recomendaciones en función de los antecedentes de resistencia del hormigón.}
\label{tab:conclusiones}
\end{table}

Adicionalmente, se puede realizar un análisis considerando el total de muestras. El ensayo se considera satisfactorio si se cumple que:

\begin{itemize}
    \item Criterio global: $f_{m} \geq f_{c} + s \cdot t$
    \item Criterio individual: $f_{i} \geq f_{0} = f_c - k_2$
\end{itemize}

Donde $s$ corresponde a la desviación estándar de la muestra y el valor $t$ se calcula a partir de los datos de la Tabla 4 de la norma \cite{NCh1998}.

La Tabla \ref{tab:conclusiones2} muestra las conclusiones obtenidas en este análisis.

\begin{table}[H]
\centering
\begin{tabular}{|c|c|p{6cm}|p{6cm}|}
\hline
\textbf{Antecedentes} &  & \textbf{Conclusiones} & \textbf{Recomendaciones} \\ \hline
$f_{m} \geq f_{c} + s \cdot t$ & $f_{i} \geq f_{0}$ & 
El hormigón cumple la resistencia especificada & \\ \hline

$f_{m} < f_{c} + s \cdot t$ & $f_{i} \geq f_{0}$ & 
El hormigón no cumple la resistencia especificada & 
Informar a los proyectistas estructurales y considerar las penalizaciones establecidas en el contrato y sus documentos anexos. \\ \hline

 & $f_{i} < f_{0}$ & 
El hormigón no cumple la resistencia especificada y cada resultado defectuoso debe ser considerado como un riesgo potencial & 
Adoptar las medidas indicadas en A.4 \\ \hline
\end{tabular}
\caption{Conclusiones y recomendaciones considerando $f_{m}$, $f_{c}$, $s$, $t$ y $f_{0}$.}
\label{tab:conclusiones2}
\end{table}

Finalmente, si se dispone de al menos 10 muestras distintas, con dos o más probetas por cada una, se puede evaluar el nivel de control de los ensayos.

En primer lugar, es necesario calcular el intervalo promedio $R$:

\begin{equation}
    \bar{R} = \frac{\sum R_i}{N}
\end{equation}

Donde:

\begin{itemize}
    \item $R_i = f_{max,i} - f_{min,i}$ es el rango de cada muestra.
    \item $N$ es el número total de muestras.
\end{itemize}

A continuación, se debe calcular la desviación normal de los ensayos:

\begin{equation}
    s_1 = \frac{\bar{R}}{d_2}
\end{equation}

Para el valor de $d_2$, considerar los datos de la Tabla \ref{tab:d2}.

\begin{table}[H]
\centering
\begin{tabular}{|c|c|}
\hline
\textbf{Número de probetas, $n_{0}$} & \textbf{$d_{2}$} \\ \hline
2 & 1,128 \\ \hline
3 & 1,693 \\ \hline
4 & 2,059 \\ \hline
5 & 2,326 \\ \hline
6 & 2,534 \\ \hline
\end{tabular}
\caption{Valores de $d_{2}$ en función del número de probetas $n_{0}$.}
\label{tab:d2}
\end{table}

Finalmente, se calcula el coeficiente de variación de ensayo:

\begin{equation}
    V_1 = \frac{s_1}{f_m} \cdot 100\%
\end{equation}

La Tabla \ref{tab:conclusiones3} muestra las conclusiones y recomendaciones en función del coeficiente de variación obtenido.

\begin{table}[H]
\centering
\begin{tabular}{|c|c|}
\hline
\textbf{$V_{1}$ (\%)} & \textbf{Nivel de control de ensayos} \\ \hline
$0 \leq V_{1} \leq 3,0$ & Excelente \\ \hline
$3,0 < V_{1} \leq 4,0$ & Muy bueno \\ \hline
$4,0 < V_{1} \leq 5,0$ & Bueno \\ \hline
$5,0 < V_{1} \leq 6,0$ & Aceptable \\ \hline
$6,0 < V_{1}$ & Deficiente \\ \hline
\end{tabular}
\caption{Clasificación del nivel de control de ensayos según el valor de $V_{1}$.}
\label{tab:conclusiones3}
\end{table}

\subsubsection{Análisis de testigos}

La norma \cite{NCh1171-1-2001} establece los requisitos de análisis para la extracción de testigos. Se deben cumplir las siguientes condiciones:

\begin{itemize}
    \item El diámetro del testigo debe ser mayor que el doble del diámetro máximo del agregado.
    \item La altura debe ser tal que la esbeltez (altura/diámetro) esté entre 1,0 y 2,0.
    \item La resistencia del hormigón de extracción debe ser mayor a 8 MPa y tener al menos 14 días.
    \item Además, los límites físicos de extracción están especificados por la norma.
\end{itemize}

\subsection{Resultados}

Los datos de entrada corresponden a un hormigón G55 con una fracción defectuosa del 20\%. Además, se cuenta con las siguientes muestras y probetas (Tabla \ref{tab:datosentrada}).

\begin{table}[H]
\centering
\begin{tabular}{|c|c|c|c|}
\hline
\textbf{Correlativo} & \textbf{7 días [MPa]} & \textbf{28 días [MPa]} & \textbf{28 días [MPa]} \\ \hline
1  & 41.45 & 52.80 & 54.60 \\ \hline
2  & 40.20 & 57.11 & 56.60 \\ \hline
3  & 43.50 & 52.35 & 55.18 \\ \hline
4  & 41.00 & 55.25 & 54.20 \\ \hline
5  & 40.00 & 54.48 & 53.82 \\ \hline
6  & 41.00 & 55.30 & 52.08 \\ \hline
7  & 43.50 & 51.00 & 53.82 \\ \hline
8  & 41.00 & 48.40 & 49.90 \\ \hline
9  & 40.00 & 49.50 & 45.60 \\ \hline
10 & 40.20 & 52.56 & 47.88 \\ \hline
11 & 43.50 & 57.11 & 56.60 \\ \hline
12 & 40.00 & 52.35 & 52.08 \\ \hline
13 & 41.00 & 55.25 & 53.82 \\ \hline
14 & 43.50 & 54.48 & 53.82 \\ \hline
15 & 41.00 & 55.30 & 57.79 \\ \hline
16 & 40.00 & 58.15 & 56.60 \\ \hline
17 & 40.20 & 55.25 & 55.18 \\ \hline
18 & 43.50 & 52.56 & 53.82 \\ \hline
19 & 41.00 & 48.40 & 45.68 \\ \hline
20 & 40.00 & 49.60 & 47.70 \\ \hline
\end{tabular}
\caption{Resultados de resistencia a compresión de probetas a 7 y 28 días.}
\label{tab:datosentrada}
\end{table}

Luego, se obtienen los resultados de cumplimiento que se presentan en la Tabla \ref{tab:cumplimiento}.

\begin{table}[H]
\centering
\begin{tabular}{|c|c|c|}
\hline
\textbf{Probeta} & \textbf{Cumple $f_c - k_2$} & \textbf{Cumple $f_3 \geq f_c + k_1$} \\ \hline
1  & Sí & Sí \\ \hline
2  & Sí & Sí \\ \hline
3  & Sí & Sí \\ \hline
4  & Sí & Sí \\ \hline
5  & Sí & No \\ \hline
6  & Sí & No \\ \hline
7  & Sí & No \\ \hline
8  & No & No \\ \hline
9  & No & No \\ \hline
10 & Sí & No \\ \hline
11 & Sí & Sí \\ \hline
12 & Sí & Sí \\ \hline
13 & Sí & Sí \\ \hline
14 & Sí & Sí \\ \hline
15 & Sí & Sí \\ \hline
16 & Sí & Sí \\ \hline
17 & Sí & No \\ \hline
18 & Sí & No \\ \hline
19 & No & -  \\ \hline
20 & No & -  \\ \hline
\end{tabular}
\caption{Resultados de cumplimiento de las condiciones estadísticas de resistencia del hormigón.}
\label{tab:cumplimiento}
\end{table}

Además, el criterio $f_m \geq f_c + s t$ (esto es, $f_m \geq f_c + s \cdot t$) no se cumple. Por lo tanto, corresponde adoptar las medidas indicadas en el apartado A.4 de la norma \cite{NCh1998}.

Se concluye que:

\begin{itemize}
    \item Existe riesgo estructural y el hormigón afectado debe ser sometido a análisis.
    \item Se debe comprobar la validez del ensayo y, adicionalmente, extraer testigos de hormigón endurecido para identificar zonas comprometidas y evaluar la resistencia real en obra, de acuerdo con \citep{NCh1171-1-2001,NCh1171-2-2001}.
    \item Es necesario reforzar el control de calidad en obra para evitar este tipo de incumplimientos, asegurando condiciones adecuadas de confección, transporte y curado del hormigón \citep{NCh170-2016}.
\end{itemize}

De acuerdo con la \citep{NCh1998}, el nivel de control se evalúa cuando existen a lo menos 10 muestras con 2 o más probetas por muestra. Se calculan: (i) el rango medio $\overline{R}$ entre probetas compañeras, (ii) la desviación normal de ensayos $s_1=\overline{R}/d_2$ (con $d_2$ según número de probetas por muestra) y (iii) el coeficiente de variación del ensayo $V_1=100\,s_1/f_m$. La norma clasifica el control como \emph{Excelente}, \emph{Muy bueno}, \emph{Bueno}, \emph{Aceptable} o \emph{Deficiente} en función de $V_1$ (con umbrales 3\%, 4\%, 5\% y 6\%).

Con esto, el nivel de control del ensayo resultó \textit{Excelente} ($V_1 = 3\%$), lo que refleja baja variabilidad entre probetas y un adecuado proceso de muestreo. No obstante, la resistencia promedio no cumple el valor exigido, lo que confirma que la causa principal del incumplimiento se debe a la resistencia real del lote y no a errores de ensayo. Algunos factores que pueden mejorar o empeorar el nivel de control detectado pueden ser la capacitación del personal, mal calibrado de los equipos, el uso de materiales de distintas calidades y condiciones ambientales variables.

Al modificar la fracción defectuosa, no se observa cambio al aumentarla, dado que ya se considera el límite superior (20\%). En cambio, al reducirla disminuye el número de muestras aprobadas, pues los factores estadísticos $k$ aumentan, incrementando la exigencia de los criterios de aceptación y reduciendo el número de resultados válidos.

Por ejemplo, con 26 muestras iniciales aprobadas bajo el 20\% de defectuosos, si se restringe a 5\%, solo 15 probetas cumplen la condición. Esta sensibilidad demuestra la relevancia de definir correctamente la fracción defectuosa en la especificación de obra, ya que influye directamente en la aceptación o rechazo de lotes y en la necesidad de aplicar medidas correctivas adicionales \citep{NCh1998}.

Cabe añadir que los tres testigos extraídos cumplen con los requisitos geométricos y de preparación de la NCh1171/1: diámetro de 150 mm suficiente frente al tamaño máximo del árido, esbeltez entre 1,20 y 1,23 (dentro del rango 1,0–2,0) y caras rectificadas como corresponde a un hormigón de grado $\ge$H45. Con edades de 45 días y resistencias medidas entre 51,5 y 53,5 MPa, al aplicar la corrección por esbeltez ($k_1\simeq0,912$) se obtienen valores equivalentes de 46,9–48,8 MPa, con promedio 47,8 MPa. Comparados con la resistencia de diseño $f_c=55$ MPa, los criterios del Anexo A de la NCh1998 se cumplen, ya que cada resultado supera $0,75f_c=41,25$ MPa y el promedio excede $0,85f_c=46,75$ MPa; por lo tanto, la zona investigada puede considerarse aceptable. 

\subsubsection{Especificación Prescriptiva}

Considerando los datos entregados y el Taller 1, se realizó la siguiente especificación técnica prescriptiva.

\subsubsection*{1. Supuestos}
\begin{itemize}
    \item Elemento: fundaciones y muros corrientes en obra de edificación.
    \item Clase de exposición: C0 (no agresiva).
    \item Control en obra: Bueno–Muy bueno (dosificación en peso, laboratorio en faena, control permanente).
    \item Fracción defectuosa admisible: 20\%, con criterios $k_1$ y $k_2$ según NCh1998.
    \item En caso de no conformidad, verificación mediante testigos según NCh1171/1 y NCh1171/2.
\end{itemize}

\subsubsection*{2. Materiales}
\begin{itemize}
    \item Cemento: Portland (Melón Extra) conforme a normas chilenas vigentes.
    \item Agua: potable, limpia, libre de aceites, sulfatos y cloruros nocivos \citep{INN2022_NCh1498}.
    \item Áridos: naturales triturados, gradación continua conforme la norma \citep{NCh163_Of2013}; $D_{máx} = 19$ mm. 
    \item Propiedades de referencia: densidades 2,6–2,8 g/cm$^3$ y absorciones 1,2–1,3\%.
    \item Aditivos: Superplastificante para mejorar la trabajabilidad (HRWR).
    \item Aire incorporado: 1–2\%.
\end{itemize}

\subsubsection*{3. Proporciones por m$^3$ de hormigón}
\begin{itemize}
    \item Cemento: 380–420 kg
    \item Agua: 150–165 kg
    \item Relación w/c objetivo: 0,38–0,42
    \item Árido grueso total: 950–1050 kg (60–65\% del total)
    \item Árido fino: 650–750 kg (35–40\% del total)
    \item Aditivo HRWR: dosificar para reducción de agua 20–30\%
    \item Aire: 1–2\%
\end{itemize}
La granulometría combinada debe cumplir con la Curva Tarántula optimizada en el Taller 1, asegurando cohesión y bombeabilidad.

\subsubsection*{4. Consistencia y colocación}
\begin{itemize}
    \item Asentamiento (slump): 75–100 mm (NCh170).
    \item Producción y transporte: dosificación en peso, corrección por humedad y absorción de áridos en planta.
    \item Colocación: caída libre $\leq 1,5$ m; evitar juntas frías; vibrado interno $\geq 8000$ vpm.
\end{itemize}

\subsubsection*{5. Curado}
\begin{itemize}
    \item Iniciar $\leq 30$ minutos tras la colocación.
    \item Mantener al menos 7 días continuos con mantas húmedas, riego o membranas.
\end{itemize}

\subsubsection*{6. Control de calidad}
Según NCh1998:
\begin{align}
    f_3 &\geq f_c + k_1 \\
    f_i &\geq f_c - k_2
\end{align}
Para fracción defectuosa del 20\% y hormigón H55: $k_1 = 1,5$ MPa, $k_2 = 5,5$ MPa.  

\noindent Nivel de control del ensayo:
\begin{equation}
    s_1 = \frac{\bar{R}}{d_2}, \qquad V_1 = 100 \frac{s_1}{f_m}
\end{equation}
En este caso, $V_1 \approx 3\% \Rightarrow$ nivel excelente.

\subsubsection{Especificación por Desempeño}

\subsubsection*{2. Requisitos de desempeño}

\begin{itemize}
    \item Resistencia mecánica:
    \begin{itemize}
        \item Resistencia a compresión a 28 días: $f'c = 55$ MPa.
        \item Ensayo: compresión de cilindros $150 \times 300$ mm \citep{NCh1037_Of2009}.
    \end{itemize}

    \item Durabilidad frente a penetración de cloruros:
    \begin{itemize}
        \item Carga eléctrica pasante $\leq 1500$ C (durabilidad alta).
        \item Ensayo: ASTM C1202 (RCPT, Rapid Chloride Permeability Test) \citep{ASTM_C1202_2012}.
    \end{itemize}

    \item Retracción por secado:
    \begin{itemize}
        \item Deformación de retracción $\leq 600$ $\mu\varepsilon$ a 90 días.
        \item Ensayo: \citep{ASTM_C157_2014} (longitud de prismas $75 \times 75 \times 285$ mm).
    \end{itemize}

    \item Fisuración temprana por contracción plástica:
    \begin{itemize}
        \item No presentar fisuras superiores a 0,2 mm en los primeros 7 días bajo exposición ambiental normal.
        \item Ensayo: \citep{ASTM_C1579_2013} (placas restringidas).
    \end{itemize}

    \item Módulo de elasticidad:
    \begin{itemize}
        \item $E_c \geq 35$ GPa a 28 días.
        \item Ensayo: \citep{ASTM_C469_2002}.
    \end{itemize}

    \item Contenido de aire y permeabilidad:
    \begin{itemize}
        \item Contenido de aire total: 1–3\%.
        \item Ensayo: \citep{ASTM_C231_2014} (método de presión).
    \end{itemize}

    \item Trabajabilidad:
    \begin{itemize}
        \item Asentamiento (slump): 75–100 mm.
        \item Ensayo: \citep{NCh1019_Of2009}.
    \end{itemize}
\end{itemize}


Estas especificaciones y control de calidad que se analizó para este taller, se puede aplicar en una obra real, como un edificio de oficinas, un puente o un pavimento urbano, y ver si el hormigón cumple con las especificaciones de resistencia, durabilidad y desempeño establecidas en el proyecto. En el caso de un edificio de oficinas, los valores de resistencia a compresión y propiedades mecánicas se utilizan para calcular la capacidad de resistencia de cada componente de la estructura, asegurando que se cumplen los factores de seguridad. En un puente, la durabilidad y la resistencia al ingreso de cloruros son fundamentales para mantener la vida útil de los elementos expuestos al ambiente. En un pavimento urbano, la retracción y el control de la fisuración se deben tomar en cuenta para evitar grietas prematuras que afecten el desempeño del pavimento.

Si el hormigón no cumple las especificaciones de diseño, como ingeniero  se deben tomar decisiones en base a las normas actuales. Primero se debe confirmar el incumplimiento mediante un análisis de control de calidad. Posteriormente se procede a la extracción de testigos en obra para obtener la resistencia del material utilizado. Con estos resultados se debe realizar una evaluación estructural comparando las resistencias obtenidas con las solicitadas en el diseño y recalculando la seguridad de los elementos afectados.

A partir de esta evaluación se deben tomar medidas correctivas. Si los resultados son aceptables dentro de márgenes de seguridad, la estructura puede mantenerse con los debidos mantenimientos. Si los resultados están al límite de lo aceptable, se recomienda poner reforzamientos como fibras, postensado adicional o recubrimientos protectores. Si los resultados no cumplen lo solicitado, se debe optar por la demolición y reconstrucción de los elementos defectuosos. Finalmente, se puede mandar el análisis y control de fallas a la empresa encargada del hormigón.



\subsection{Discusión}

\begin{enumerate}
    \item El proceso de hidratación del cemento es fundamental para el desarrollo de la resistencia a compresión del hormigón, pues genera las reacciones entre el agua y los compuestos del cemento, consolidando la matriz. A los 7 días, suele alcanzarse un 60--70\% de hidratación, lo que explica resistencias significativas; a los 28 días se obtiene la resistencia característica, y a los 90 días la hidratación residual incrementa la densificación y la durabilidad. Factores externos como temperatura y humedad afectan la cinética: temperaturas altas aceleran la hidratación pero pueden inducir microfisuras y pérdidas de resistencia a largo plazo; temperaturas bajas la retardan. Un curado con adecuada humedad asegura continuidad de la hidratación; en cambio, baja humedad ralentiza el proceso, aumenta la porosidad y reduce resistencia y durabilidad \citep{NCh170-2016}.
    
    \item En términos de sustentabilidad, los aditivos más frecuentes (reductores de agua, superplastificantes e incorporadores de aire) mejoran la trabajabilidad, aumentan la resistencia y permiten disminuir la relación agua/cemento, aunque con un impacto ambiental asociado a su producción. Por ejemplo, se ha reportado que superplastificantes a base de naftaleno–formaldehído pueden aportar del orden de 50--80 kg CO$_2$ eq/ton de aditivo \citep{Habert2011,Flower2007}. En la mezcla final, este aporte suele ser menor al 1\% de la huella total, y sus beneficios en reducción de agua y mayor durabilidad tienden a compensarlo al extender la vida útil. Las especificaciones prescriptivas podrían limitar dosificaciones máximas; las de desempeño, en cambio, regulan resultados (p.ej., resistencia a 28 días), habilitando su uso si se cumple el criterio de sustentabilidad \citep{NCh170-2016}.
    
    \item Los ciclos hielo–deshielo dañan el hormigón por expansión del agua en poros capilares al congelarse, generando presiones internas, microfisuras y pérdida progresiva de resistencia y durabilidad. Una especificación prescriptiva puede exigir contenido mínimo de aire incorporado (p.ej., 4--6\%) para mejorar la resistencia al congelamiento; una de desempeño demandará superar ensayos de durabilidad específicos. Ambos enfoques mitigan el daño; el enfoque por desempeño favorece mayor flexibilidad y adaptación a condiciones locales \citep{Powers1954,MehtaMonteiro2014}.
    
    \item La microestructura (distribución de poros, densidad de la matriz y microfisuras) influye directamente en la resistencia y durabilidad. Para su evaluación se emplean, entre otras, las siguientes técnicas:
    \begin{itemize}
        \item Microscopía electrónica de barrido (SEM): observa morfología de poros, distribución de productos de hidratación y microfisuras, con información cualitativa y cuantitativa \citep{Diamond1999}.
        \item Porosimetría por intrusión de mercurio (MIP): cuantifica tamaño y volumen de poros, permitiendo correlacionar porosidad total con resistencia y durabilidad \citep{Gallucci2012}.
    \end{itemize}
    
    \item Para este caso, una especificación por desempeño para exposición severa debe asegurar: (i) \textit{durabilidad} frente a sulfatos (p.ej., expansión $<0{,}05\%$ a seis meses en ensayos acelerados), (ii) \textit{resistencia a compresión} $\geq 40$ MPa a 28 días, (iii) \textit{impermeabilidad} (penetración de agua $\leq 20$ mm bajo presión), (iv) \textit{resistencia química} con pérdidas de resistencia $<20\%$ frente a soluciones de sulfatos, y (v) \textit{control de fisuración} con anchos $\leq 0{,}2$ mm. Complementariamente, modelos de vida útil que consideren transporte iónico y carbonatación permiten proyectar durabilidad $\geq 50$ años \citep{NCh170-2016,ConcreteSociety2006}.
    
    \item Una referencia internacional de especificación por desempeño es la norma europea EN 206, que define requisitos de durabilidad y resistencia por clase de exposición. Un caso emblemático es el Puente de Oresund (Dinamarca–Suecia), con ambiente marino severo: se exigió resistencia característica mínima de 50 MPa, límites a la penetración de cloruros y desempeño satisfactorio en ciclos hielo–deshielo. La especificación es por desempeño porque no fija dosificación ni materiales, sino criterios medibles. La verificación incluyó ensayos acelerados de cloruros, resistividad eléctrica y compresión a 28 y 90 días, garantizando vida útil y seguridad estructural en ambiente agresivo \citep{EN206,ConcreteSociety2006}.
\end{enumerate}






