\section{Desarrollo}

\subsection{Metodología}

En primer lugar se realizo un analisis estadistico de la resistencia mecanica, basado en la norma \cite{NCh1998}, la cual tiene como objetivo determinar la conformidad de los resultados de la resistencia real a comprecion respecto a la especificada.

Se requieren los siguientes requicitos base para poder aplicar la norma:

\begin{itemize}
    \item Toma de muestras al azar
    \item Toma de muestras entre el 10\% y 90\% de decarga del camion mixer
    \item Tamaño superior a 1,5 veces nesesario
    \item Generar al menos 3 probetas por muestra, 1 a 7 dias y el resto a 28 dias.
\end{itemize}

La tabla \ref{tab:muestreo} detalla el numero y volumen de muestras segun la procedencia del hormigon.

\begin{table}[H]
\centering
\begin{tabular}{|l|c|c|}
\hline
\textbf{Procedencia del hormigón} & \multicolumn{2}{c|}{\textbf{Volumen de hormigón de la obra, m$^3$}} \\ \hline
                                  & $> 250$ & $\leq 250$ \\ \hline
\textbf{Fabricado en obra}        &         &           \\ \hline
Volumen máximo de hormigón por muestra & 100     & 50        \\ \hline
Número mínimo de muestras              & 5       & 3         \\ \hline
\textbf{De central hormigonera}   &         &           \\ \hline
Volumen máximo de hormigón por muestra & 150     & 75        \\ \hline
Número mínimo de muestras              & 5       & 3         \\ \hline
\end{tabular}
\caption{Requisitos de muestreo según la procedencia y volumen de hormigón.}
\label{tab:muestreo}
\end{table}

Ademas, el plan de muestreo debe ser especificado en la norma de diseño o especificacion de la obra, de lo contrario considerar que;

\begin{itemize}
    \item Pavimentos: testigo cada 1000 m$^2$ con minimo 3.
    \item General: testigos por zona
\end{itemize}

\subsubsection{Analisis estadistico resistencia mecanica}

Si se tiene un N mayor a 10 muestras, se puede considerar la resistencia de cada lote como satisfactoria si se cumple que:

\begin{itemize}
    \item Global: $f_3 \geq f_c + k_1$
    \item Individual: $f_1 \geq f_o = f_c - k_2$
\end{itemize}

Donde, $f_3$ corresponde a la resistencia media de tres muestras consecutivas a 28 dias, $f_c$ es la resistencia especificada, $f_1$ es la resistencia individual de cada muestra y $f_o$ es la resistencia inferior para cada $f_i$ a 28 dias.

Para los valores de $k_1$ y $k_2$ ver tabla \ref{tab:valoresk}

\begin{table}[H]
\centering
\begin{tabular}{|c|c|c|c|c|c|}
\hline
\textbf{Fracción defectuosa aceptada, \%} &  & \multicolumn{4}{c|}{\textbf{Grado de hormigón}} \\ \hline
 &  & H5 & H10 & H15 & H20 o superior \\ \hline
\multirow{2}{*}{5}  & $k_{1}$ & 0,3 & 0,5 & 0,8 & 1,0 \\ \cline{2-6}
                    & $k_{2}$ & 0,6 & 1,2 & 1,9 & 2,5 \\ \hline
\multirow{2}{*}{10} & $k_{1}$ & 0   & 0   & 0   & 0   \\ \cline{2-6}
                    & $k_{2}$ & 0,9 & 1,7 & 2,6 & 3,5 \\ \hline
\multirow{2}{*}{20} & $k_{1}$ & -0,4 & - 0.7   & -1,1 & - 1,5 \\ \cline{2-6}
                    & $k_{2}$ & 1,4 & 2,7 & 4,1 & 5,5 \\ \hline
\end{tabular}
\caption{Valores de $k_{1}$ y $k_{2}$ según la fracción defectuosa aceptada y el grado de hormigón.}
\label{tab:valoresk}
\end{table}

Finalmente, la tabla \ref{tab:concluciones} muestra las cncluciones que se pueden lograr del ensayo.

\begin{table}[H]
\centering
\begin{tabular}{|c|c|p{6cm}|p{6cm}|}
\hline
\textbf{Antecedentes} &  & \textbf{Conclusiones} & \textbf{Recomendaciones} \\ \hline
$f_{3} \geq f_{c} + k_{1}$ & $f_{i} \geq f_{0}$ & 
El hormigón cumple la resistencia especificada & \\ \hline

$f_{3} < f_{c} + k_{1}$ & $f_{1} \geq f_{0}$ & 
El hormigón no cumple la resistencia especificada & 
Informar a los Proyectistas Estructurales y considerar las penalizaciones establecidas en el Contrato y sus Documentos anexos. \\ \hline

 & $f_{i} < f_{0}$ & 
El hormigón no cumple la resistencia especificada y cada resultado defectuoso debe ser considerado como un riesgo potencial & 
Adoptar las medidas indicadas en A.4 \\ \hline
\end{tabular}
\caption{Conclusiones y recomendaciones en función de los antecedentes de resistencia del hormigón.}
\label{tab:concluciones}
\end{table}

