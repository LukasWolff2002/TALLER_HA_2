\section{Introducción}

El control de calidad del hormigón en obra es un proceso fundamental para garantizar la seguridad estructural, la durabilidad y el cumplimiento de las especificaciones técnicas establecidas en los proyectos de construcción. Debido a que el hormigón es un material compuesto cuya resistencia y desempeño dependen tanto de su dosificación como de las condiciones de colocación, compactación y curado, es muy importante tener procedimientos normalizados que permitan verificar su comportamiento en diferentes etapas. En Chile, este proceso se encuentra regulado por la NCh170.Of2016, que establece los requisitos generales del hormigón, la NCh1998.Of1989, que define la evaluación estadística de la resistencia mecánica, y la NCh1171/1.Of2001 y NCh1171/2.Of2001, que norman la extracción y evaluación de testigos de hormigón endurecido.

Este informe busca que se apliquen las normas para evaluar la resistencia a compresión de probetas y testigos de hormigón, se analicen la influencia de parámetros como la fracción defectuosa y el nivel de control de ensayos, y se comprendan las diferencias entre especificaciones prescriptivas y por desempeño. De esta forma, se ve la importancia de implementar medidas correctivas cuando los resultados no cumplen con lo requerido. El taller también enfatiza el rol del curado y de la microestructura en el desarrollo de la resistencia, relacionando los aspectos técnicos con la práctica, además de la vida útil de las estructuras.


